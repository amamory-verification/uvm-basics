\documentclass[a4paper,BCOR7mm,12pt,pointlessnumbers,bibtotoc]{scrartcl}

\usepackage{amsmath,epsfig}
\usepackage{amssymb,amsfonts}
\usepackage{color}
\usepackage{array,booktabs}
\usepackage{graphicx}
\usepackage{caption}
\usepackage[hypcap=true,labelformat=simple]{subcaption}
\renewcommand{\thesubfigure}{(\alph{subfigure})}
\usepackage{tikz}
\usetikzlibrary{arrows,automata}
\usepackage{listings}
\usepackage{hyperref}
\usepackage{enumitem}

\newcolumntype{C}[1]{>{\centering\arraybackslash}p{#1}} % centering column type with fixed width
\newcolumntype{R}[1]{>{\raggedleft\arraybackslash}p{#1}} % right aligned column type with fixed width
\newcolumntype{L}[1]{>{\raggedright\arraybackslash}p{#1}} % left aligned column type with fixed width

\newcommand{\ceil}[1]{\left\lceil #1 \right\rceil} %\left\lceil #1 \right\rceil

\begin{document}
%\maketitle
\begin{center}
\Large Efficient Booth Array Multiplier for Xilinx FPGAs\\[0.4cm]
\large Martin Kumm \\[0.5cm]
\large \today \\[0.5cm]
\end{center}

\section{Introduction}

This IP core provides a resource efficient implementation of a Booth Array Multiplier for Xilinx FPGAs. For details about the architecture see \cite{kaz15}.

\section{Interface}

The top level entity is found in \verb|mult_booth_array.vhd|. The generics as well as the portare described in Table~\ref{tab:generics} and Table~\ref{tab:port}, respectively.

\begin{table}[!h]
	\renewcommand{\arraystretch}{1.1}
	\caption{Description of the generics}
	\label{tab:generics}
	\centering
	\begin{tabular}{lccL{7cm}}
	  \toprule
	  Generic & Type & Default & Description\\
    \cmidrule(rl){1-1} \cmidrule(rl){2-2} \cmidrule(rl){3-3} \cmidrule(rl){4-4}
    \verb|word_size_a|    & integer & 8     & Input word size operand $A$\\
    \verb|word_size_b|    & integer & 8     & Input word size operand $B$\\
    \verb|sync_in_out|    & boolean & false & If true, registers are placed at inputs and outputs (for timing results)\\
    \verb|use_pipelining| & boolean & true  & If true, the multiplier is internally pipelined (highly recommented)\\
    \bottomrule
   \end{tabular}
\end{table}

\begin{table}[!h]
	\renewcommand{\arraystretch}{1.1}
	\caption{Description of the port}
	\label{tab:port}
	\centering
	\begin{tabular}{lcccL{5cm}}
	  \toprule
	  Generic & Direction & Type & Word Size & Description\\
    \cmidrule(rl){1-1} \cmidrule(rl){2-2} \cmidrule(rl){3-3} \cmidrule(rl){4-4} \cmidrule(rl){5-5}
    \verb|clk_i| & in  & \verb|sl|    & 1                              & Clock input (used when \verb|use_pipelining=true| or \verb|sync_in_out=true|)\\
    \verb|rst_i| & in  & \verb|sl|    & 1                              & Reset input (used when \verb|use_pipelining=true| or \verb|sync_in_out=true|)\\
    \verb|ce_i|  & in  & \verb|sl|    & 1                              & Clock enable input (used when \verb|use_pipelining=true| or \verb|sync_in_out=true|)\\
    \verb|a_i|   & in  & \verb|slv|   & \verb|word_size_a|             & Input operand $A$\\
    \verb|b_i|   & in  & \verb|slv|   & \verb|word_size_b|             & Input operand $B$\\
    \verb|p_o|   & out & \verb|slv|   & \verb|word_size_a+word_size_b| & Product output $P=A \times B$\\
    \bottomrule
   \end{tabular}
\end{table}


\section{Simulation \& Test}

For simulation and test, the testbench (\verb|tb_mult_booth_array.vhd|) was created which uses a random number generator together with assert statements to verify the designs (against a naive VHDL multiplication). 

\bibliographystyle{alpha}
\bibliography{booth_array_multiplier}

\end{document}


 
